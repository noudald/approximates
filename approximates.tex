\documentclass[10pt]{article}

\usepackage[english]{babel}
\usepackage{
    amsmath,
    amssymb,
    amsthm,
    bm,
    fullpage,
}

\theoremstyle{definition}
\newtheorem{theorem}{Theorem}[section]
\newtheorem{example}{Example}[section]

\begin{document}

\title{Approximates for the working mathematician}
\author{Noud Aldenhoven}
\date{\today}
\maketitle

\section{Log and exponential}

The exponential constant is approximated as
\begin{equation*}
    e \approx 2.71828
\end{equation*}

\begin{align*}
    \log(1) &= 0,               & e^0 &= 1 \\
    \log(2) &\approx 0.6931,    & e^{0.69314} &\approx 2 \\
    \log(e) &= 1,               & e^{1} &= e \\
    \log(10) &\approx 2.3025    & e^{2.30258} &\approx 10,
\end{align*}

Note that the Tayler expension of $e^x$ is
\begin{equation*}
    e^x = 1 + x + \frac{x^2}{2!} + \frac{x^3}{3!} + ...,
\end{equation*}
So for small numbers $x \approx 0$ we have $e^x = 1 + x$.

\begin{example}
    This example is from Richard Feynman's autobiography ``Surely You're Joking Mr. Feynman''.
    How to approximate $e^{3.3}$ fast?
    Note that
    \begin{equation*}
        e^{3.3} = e^{2.30258} e^{1} e^{-0.00258}
            \approx 10 \times e \times (1 - 0.00258)
            = 10 \times e - 0.0258 \times e
    \end{equation*}
    So without effort we can approximate the $e^{3.3} \approx 10 e = 27.1$.
    What's left to do is correct for the overestimation.
    \begin{equation*}
        0.00258 \times e
            \approx 0.0258 \times 2.71828
            \approx 0.0701
    \end{equation*}
    Hence the final approximation is $e^{3.3} \approx 27.1828 - 0.0701 = 27.1127$.
    The correct answer is $e^{3.3} = 27.1126389...$.
\end{example}


\end{document}
